\section{Altri capitolati}
\subsection {C1 – Actorbase: A NoSQL Based on the Actor model}
\subsubsection{Valutazione Generale}
Il capitolato richiede l'implementazione di ACTORDB ovvero di un database NoSQL di tipo \textit{key-value} che utilizzi il modelli ad attori, un modello matematico di esecuzione concorrente di un programma nel quale le primitive di elaborazione concorrente sono individuate negli attori. Gli attori sono oggetti reattivi che eseguono delle istruzioni in risposta a messaggi e che inviano messaggi di risposta al termine dell'esecuzione. In particolare si chiede di implementare all'interno del DB l'inserimento, la cancellazione e l'aggiornamento, e infine la definizione di un \textit{domain specific language} (DSL\G).\\
Il gruppo di lavoro non ha dimostrato particolare interesse per il suddetto capitolato, inoltre al momento della scelta non era più disponibile per esaurimento del numero massimo dei vincitori.

\subsection {C2 – CLIPS: Communication \& Localisation with Indoor Positioning Systems}
\subsubsection{Valutazione generale}
Il capitolato pone come obiettivo la ricerca e la sperimentazione di nuovi scenari per l'implementazione della navigazione \textit{indoor} applicata a più ambiti. Alla base di questo progetto sta l'IPS\G\, (\textit{Indoor Positioning System}) il principio che permette di localizzare oggetti e persone all'interno di edifici usando onde radio, campi magnetici e segnali acustici. \\
Il gruppo ha dimostrato un particolare interesse verso questa proposta dal momento che tratta una tecnologia in forte espansione e che ha una stretta correlazione con il mondo mobile, in particolare con gli strumenti di socializzazione moderni. Inoltre l'azienda si è dimostrata fin da subito disponibile a voler sostenere e supportare i vincitori con incontri e delucidazioni, in aggiunta a materiale tecnologico da mettere a disposizione per il \textit{team}.\\
Tuttavia il suddetto capitolato non propone lo sviluppo effettivo di un software specifico, ma si limita a richiedere uno studio approfondito di nuove sperimentazioni della tecnologia IPS\G\,.

\subsubsection{Fattori di rischio}
\begin{itemize}
\item Difficoltà nel trovare un nuovo ambito di studio che possa risultare originale e interessante;
\item Poca conoscenza della materia di studio e delle realtà che lo sviluppano;
\item Scarso utilizzo delle conoscenze sviluppate ed acquisite negli anni di studio.
\end{itemize}

\subsection{C3 – UMAP: un motore per l'analisi predittiva in ambiente Internet of Things}
\subsubsection{Valutazione Generale}
Il capitolato chiede di creare un algoritmo predittivo in grado di analizzare i dati provenienti da oggetti inseriti in diversi contesti e fornire delle previsioni su possibili guasti, interazioni con nuovi utenti e identificare dei pattern di comportamento degli utenti per prevedere le azioni degli stessi su altri oggetti o contesti. In particolare chiede la progettazione di una piattaforma che consti di tre parti: una console web per l'amministrazione, una console web di amministrazione per la singola azienda e servizi di \textit{Web Restful JSON\G\,} interrogabili.\\
Il team non si è dimostrato particolarmente interessato alle tecnologie da utilizzare in questo capitolato, inoltre al momento della scelta era già stato raggiunto il numero massimo di vincitori.

\subsection{C4 – MaaS: MongoDB as an admin Service}
\subsubsection{Valutazione generale}
Il capitolato in esame si occupa della realizzazione di un servizio web dedicato alla gestione di grandi quantità di dati attraverso l'implementazione di MaaP, una piattaforma di amministrazione basata su MongoDB e sviluppata da studenti di informatica dell'Università di Padova per il progetto di Ingegneria del software del 2013.\\
La documentazione fornita è esaustiva e completa: vengono forniti svariati esempi e viene data una descrizione accurata di tutti i termini presentati nel capitolato. Lo scopo del progetto è pertanto ben definito e risulta ben chiaro l'ambito su cui andrebbe a lavorare il gruppo. E stato apprezzato il fatto che l'azienda voglia proporre una piattaforma nuova per il mercato di riferimento.

\subsubsection{Fattori di rischio}
\begin{itemize}
\item Necessità di lavorare su un software giovane e realizzato da altri studenti;
\item Totale inesperienza sulle tecnologie da impiegare;
\item Licenza e \textit{copyright} vincolati all'azienda proponente;
\item Il dominio applicativo non ha attirato l'attenzione del gruppo.
\end{itemize}

\subsection{C5 – Qizzipedia: software per la gestione di questionari}
\subsubsection{Valutazione generale}
Il capitolato C5 ha per oggetto la realizzazione di un software di gestione di questionari, composto da un archivio di domande e da un sistema di test che prelevi da tale archivio i questionari specifici per l'argomento scelto dall'utente. Il sistema deve essere realizzato con tecnologie Web e si deve interfacciare all'utente tramite un browser HTML5, al fine di renderlo accessibile anche attraverso apparecchi mobile.\\
Sebbene le tecnologie da utilizzare siano molto allettanti e riguardino un ambito in continua espansione, il gruppo di lavoro ha voluto orientarsi verso una materia di studio più stimolante e che potesse portare all'acquisizione di tecnologie mai utilizzate o scarsamente conosciute dai membri. Inoltre non è stato possibile scegliere il suddetto capitolato in quanto risultava raggiunto il numero massimo di vincitori.


