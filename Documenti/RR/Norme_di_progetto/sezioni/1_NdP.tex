\section{Introduzione}

\subsection{Scopo del documento}
Questo documento definisce le norme che i membri del gruppo Stark Labs dovranno 
rispettare nello svolgimento del progetto “SiVoDiM”.
Ogni componente del gruppo è tenuto a leggere il documento e seguire le norme 
per raggiungere il miglior punto di incontro tra efficienza ed efficacia nello 
svolgimento delle attività. In questo modo viene garantita l'uniformità del 
materiale prodotto e vengono facilitate le operazioni di verifica. 
In particolare verranno specificate norme riguardanti:
\begin{itemize}
\item	Interazioni tra membri del gruppo;
\item	Comunicazione verso l'esterno;
\item	Stesura dei documenti e convenzioni tipografiche;
\item	Organizzazione dell'ambiente di lavoro;
\item	Modalità di lavoro durante le fasi del progetto;
\item	Stesura del codice.
\end{itemize}

\subsection{Scopo del progetto}
Lo scopo del progetto è lo sviluppo di un applicativo per dimostrare 
efficacemente le potenzialità del motore di sintesi vocale FA-TTS\G.  
\textbf{//eventualmente completare dopo l'incontro con Giulio}

\subsection{Glossario}
Al fine di aumentare la comprensione del testo ed evitare eventuali ambiguità, 
viene fornito un glossario (Glossario v1.0) contenente le definizioni degli 
acronimi e dei termini tecnici utilizzati nel documento. Ogni vocabolo che ha 
un riferimento contenuto nel glossario è contrassegnato dal pedice “\G “.

\subsection{Riferimenti}

\subsubsection{Normativi}
\begin{itemize}
\item Glossario v1.0.
\end{itemize}

\subsubsection{Informativi}
\begin{itemize}
\item Capitolato C1 – Actor base: A NoSQL Based on the Actor model\\
\item Capitolato C1 – Actor base: A NoSQL Based on the Actor model\\
\item Capitolato C1 – Actor base: A NoSQL Based on the Actor model\\

\end{itemize}
\newpage

